\documentclass[%
	paper=a4,
	fontsize=10pt,
	DIV11,BCOR10mm,
	numbers=noenddot,
	abstract=yes
]{scrartcl}

\usepackage[utf8]{inputenc}
\usepackage[english]{babel}
\usepackage[T1]{fontenc}
\usepackage{tgpagella}
\usepackage{microtype}


\usepackage{textcomp}
\usepackage{gensymb}


\usepackage{graphicx}


\usepackage{xfrac}
\usepackage{units}


\usepackage{amsmath}
\usepackage{amssymb}
\usepackage{amsthm}


\usepackage{mathtools}
\mathtoolsset{showonlyrefs=true}


\usepackage{scrhack} % for algorithm package
\usepackage[boxed]{algorithm}
\usepackage{algpseudocode}


\usepackage[
	style=alphabetic,
	bibstyle=alphabetic,
	maxbibnames=100,
	giveninits=true,
	useprefix=true,
	natbib=true,
	backend=biber]{biblatex}
\usepackage[strict=true]{csquotes}
\addbibresource{references.bib}


\usepackage{hyperref}
\hypersetup{
	%colorlinks,
	%citecolor=black,
	%filecolor=black,
	%linkcolor=black,
	%urlcolor=black,
	pdfauthor={Christoph Conrads},
	unicode=true,
}


% Make links jump to the beginning of figures and tables, not to the caption in
% them.
% This can also be fixed with the caption package.
\usepackage[all]{hypcap}



% Command definitions
\newcommand{\R}{\mathbb{R}}


\newtheorem{theorem}{Theorem}[section]
\newtheorem{corollary}{Corollary}[section]

\theoremstyle{definition}
\newtheorem{example}{Example}[section]
\newtheorem{definition}{Definition}[section]



\title{FizzBuzz with Fibonacci Primes}
\author{Christoph Conrads {\small \url{https://christoph-conrads.name}}}



\begin{document}

\maketitle



\section{Task}

The goal is to implement a variant of FizzBuzz \cite{Atwood2007}. Let $F(n)$ be
the $n$-th Fibonacci number. Given a positive negative integer $N$,
\begin{itemize}
	\item print \texttt{BuzzFizz} if $F(n)$ is a prime number, otherwise
	\item print \texttt{FizzBuzz} if $F(n)$ is divisible by 15, otherwise
	\item print \texttt{Fizz} if $F(n)$ is divisible by 5, otherwise
	\item print \texttt{Buzz} if $F(n)$ is divisible by 3, otherwise
	\item print $F(n)$
\end{itemize}
for all numbers $1 \leq n \leq N$.



\section{Implementation}

\begin{definition}[Fibonacci series {\cite[§2]{Hoggatt1969}}]
	The sequence
	\begin{align*}
		F(0) &\coloneqq 0, \\
		F(1) &\coloneqq 1, \\
		F(n) &\coloneqq F(n-1) + F(n-2)
	\end{align*}
	is called \emph{Fibonacci series} and the terms $F(n)$ are called
	\emph{Fibonacci numbers}.
\end{definition}

On a computer, the major problems to be solved are the representation of $F(n)$
and determinining the primality of $F(n)$.

Representing $F(n)$ is non-trivial on a computer because $F(94) \approx
10^{19.30} > 2^{64} - 1$ meaning we cannot store $F(94)$ is an unsigned
\unit[64]{bit} integer. $F(1476) \approx 10^{308.1}$ is already larger then the
maximum finite value representable with IEEE-754 double precision floats.
Consequently, the program must use a big integer implementation.

Testing for primality is difficult because $F(n)$ grows exponentially \cite[§6,
Theorem~1]{Hoggatt1969} and because there are
\[ \pi(n) \approx \frac{n}{\ln n} \]
primes below $n$ \cite[381\psq]{Knuth1998}. Hence number
sieves are probably ineffective and the memory used for storing many, large
prime numbers cannot be used for the representation of $F(n)$. Alternatively, we
can use primality tests. The Miller-Rabin primality test
\cite[395\psq]{Knuth1998} is a probabilistic test; the algorithm recognizes
prime numbers reliably whereas for composite numbers, there is a $25\%$ chance
that it may fail to recognize these numbers as a composite and this probability
holds for any natural number. Repeating the test $k$ times decreases the chance
of misclassifying a composite to $4^{-k}$. The Miller-Rabin test is very
effective in practice and used in cryptographic software, e.\,g., in Libgcrypt
in combination with other probabilistic primality tests \cite[§16.5]{libgcrypt}
or for the computation of primes for DSS \cite[\nopp C.3]{FIPS186-4}.

For Fibonacci numbers $F(n)$ it holds that $F(n)$ is a divisor of $F(kn)$, $k
\in \mathbb{N}$, $k > 1$ \cite[§7, Theorem~1]{Hoggatt1969}. Thus, $F(kn)$ must
be a composite for all $kn > 5$. Since the computational effort of the
Miller-Rabin test depends on the number of bits in the representation of $F(n)$
or $n$, respectively, and since $n \ll F(n)$, this allows us to sped up
primality checking by testing $n$ before $F(n)$ for primality. Furthermore, for
$n \leq 2^{48}$, it is sufficient to use the first thirteen prime numbers as
prime base in the Miller-Rabin test to get a true primality test
\cite{Sorensen2015}. In layman's terms, we can reliably recognize composite
numbers $n \leq 2^{48}$ with thirteen iterations of the Miller-Rabin test. For
comparison, \cite{FIPS186-4} recommends at least $k = 64$ iterations.

\printbibliography

\end{document}
